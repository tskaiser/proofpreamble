\documentclass[11pt]{article}
\usepackage[a4paper,top=1in,bottom=1in,left=1.5in,right=1.5in,landscape]{geometry}

\usepackage[utf8]{inputenc}
\usepackage[english]{babel}
\usepackage[T1]{fontenc}
\usepackage{amsmath, amssymb, amsfonts}
\usepackage{bussproofs, varwidth}
\usepackage{semantic}
\usepackage{mathabx}
\usepackage{color}
\usepackage{xspace}

\newcommand{\gt}{>}
\mathligprotect{\gt}
\newcommand{\lt}{<}
\mathligprotect{\lt}
\newcommand{\eq}{=}
\mathligprotect{\eq}

\mathlig{<}{\langle}
\mathlig{>}{\rangle}
\mathlig{!}{\downarrow}
\mathlig{|->}{\mapsto}

\newcommand{\true}{{\bf true}\xspace}
\newcommand{\false}{{\bf false}\xspace}
\newcommand{\sdash}{\sigma \vdash\xspace}

\newcommand{\IF}{{\bf if}\xspace}
\newcommand{\THEN}{{\bf then}\xspace}
\newcommand{\ELSE}{{\bf else}\xspace}
\newcommand{\WHILE}{{\bf while}\xspace}
\newcommand{\DO}{{\bf do}\xspace}
\newcommand{\REPEAT}{{\bf repeat}\xspace}
\newcommand{\UNTIL}{{\bf until}\xspace}
\newcommand{\SKIP}{{\bf skip}\xspace}

\newcommand{\hannot}[1]{\color{red} $\{#1\}$}

\EnableBpAbbreviations


\title{Example usage of proofpreamble}
\author{Thorbjørn S. Kaiser}

\setcounter{secnumdepth}{-1}

\begin{document}
\maketitle

\section{example 1: flow and indentation}

This example is meant to showcase how the prooftree
environment is built up.

\begin{prooftree}
                \AxiomC{1.1.1.1}
                \AxiomC{1.1.1.2}
                \AxiomC{1.1.1.3}
            \TrinaryInfC{1.1.1}
            \AxiomC{1.1.2}
        \BinaryInfC{1.1}
    \UnaryInfC{1}
\end{prooftree}
\begin{prooftree}
            \AxiomC{1.1.1}
                \AxiomC{1.1.2.1}
                \AxiomC{1.1.2.2}
                \AxiomC{1.1.2.3}
            \TrinaryInfC{1.1.2}
        \BinaryInfC{1.1}
    \UnaryInfC{1}
\end{prooftree}

I highly recommend using this indentation,
as it makes it easier to read.
For each fragment, UnaryInfC, BinaryInfC, TrinaryInfC,
and AxiomC they take a number of fragments above
themselves, flowing from bottom to top.

\section{example 2: labels}

This example showcases left and right labels.
As with the previous example evaluation
flows from bottom to top,
meaning the label declaration comes {\em before}
the line they belong to.

\begin{prooftree}
        \AxiomC{$<a_0, \sigma> ! n_0$}
        \AxiomC{$<a_1, \sigma> ! n_1$}
        \LeftLabel{EB-EqF}
        \RightLabel{$n_0 \neq n_1$}
    \BinaryInfC{$<a_0 = a_1, \sigma> ! \false$}
\end{prooftree}

\section{example 3: making assumptions}

This is how I make assumptions.

\begin{prooftree}
            \AxiomC{$\xi_1$}
            \noLine
        \UnaryInfC{$<a_0, \sigma> ! n_0$}
            \AxiomC{$\xi_2$}
            \noLine
        \UnaryInfC{$<a_1, \sigma> ! n_1$}
        \LeftLabel{EB-EqF}
        \RightLabel{$n_0 \neq n_1$}
    \BinaryInfC{$<a_0 = a_1, \sigma> ! \false$}
\end{prooftree}

Essentially, when you want to make an assumption, $xi_n$,
you make the ``root'' of the assumption an UnaryInfC
without a line using \textbackslash{}noLine,
followed by the assumption name in an \textbackslash{}AxiomC
fragment.

\end{document}
